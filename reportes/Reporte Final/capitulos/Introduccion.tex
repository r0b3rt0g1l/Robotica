\chapter{Introducción} \label{chap:introduccion}

En este capítulo, se explica un pequeño resumen del reporte. Los temas que se verán y explicar un poco sobre su robot sin entrar en muchos detalles.

Cabe destacar que si no quieren escribir algún capítulo, con \TeX studio pueden comentar la línea (O las líneas seleccionadas) con \menu{Ctrl} + \menu{T}, mientras que para descomentar la línea es el mismo comando.

También si presionan \menu{Ctrl} y luego hacen \menu{Clic Izquierdo} en el documento que se muestra a la derecha (después de presionar \menu{compilar y ver} o \menu{F5}), podrán abrir esa parte del código de \LaTeX. Pueden hacer lo mismo en el código para entrar a esa parte del documento o usar \menu{Ctrl} + \menu{Clic Izquierdo}.

También si quieren modificar la bibliografía, solo tienen que abrir el archivo llamado bibliografia.bib o ir al final del archivo principal, \ffile{Reporte.tex}, donde dice

\begin{latexcode}{\LaTeX}
		% Pulsa Ctrl + Clic Izquierdo en bibliografia para entrar.
		\bibliography{bibliografia}
	\end{document}
\end{latexcode}

Por último, recuerden que este formato está basado en la perfección. Y uno no puede entregar la perfección cuando queda menos de una semana para entregar la tarea y hay temas que ni completamos, así que no es necesario completarlo todo.